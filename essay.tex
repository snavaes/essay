\documentclass[a4paper, 11pt]{article}

\usepackage{natbib}
\bibliographystyle{agsm}
\setcitestyle{authoryear,open={(},close={)}}

\input{layout.tex}

\title{\textbf{Essay}\\Sub title on Essay}

\author{\textsc{Ans Vaessen}
\\{\textit{ans@nlgids.london}}}

\date{\today}

%%%%%%%%%%%%%%%%% START %%%%%%%%%%%%%%%%%%

\begin{document}
\maketitle

\begin{abstract}
This is an abstract.
\end{abstract}

\vspace{30pt} % Some vertical space between the abstract and first section

\section*{Introduction}
Is Ryan air an disruptive entrant in the flight industry?


Flying used to be an expensive mode of travel untill the low cost airlines entered the market. In 1995 two airlines came with a new concept: Low Cost Airlines. This was an innovation that changed the airline business as we know it.




See \cite{Christensen97}.
See \citep{Christensen97}.
See \citep[p. 145]{Christensen97}.


\begin{quote}
hallo
\end{quote}

\section{Ryanair and change of regulations}

In 1984 Ryanair was founded and at the beginning of the nineties started with low cost flights by copying the 'no frills' idea from Southwest airlines in the states. Calling themself the Europe’s first low fares airline \cite{Ryanair}. This meant, for example, no meals, no seat allocation and also moving to a single aircraft type and using small underused airports \citep{Diaconu} offering the lowest fares.

Helped by the new European Union deregulation to promote international trade Ryanair could expand and increase their flights in this liberalized aviation market \citep{Dianconu}.

Flights used to be regulated by countries, each country had their own national airline. British Airways (BA) for Britain and Royal Dutch Airlines (KLM) for the Netherlands, for example. Governments would agree between nations on flying schedules, amount of passengers and fares. There was no competition until the European Commission introduced a reform proces that changed the rules. "Since April 1997, any airline holding a valid Air Operators Certificate in the EU can operate on any route within the European Union, including wholly within another country, without restriction on price or capacity. As a result, European air travel has been flooded with an influx of low-cost airlines \citep{Eurocontrol}.

\cite{TiddBessant} describe regulation as a source of innovation that works as a two sided sword, restricting on one side and deregulation can create new opportunities for innovation.

\label{sec:this-is-a-section}

\section{The Innovator's Dilemma}


In his book The Innovater Dillema \cite{Christensen97} describes the innovator's dilemma and how incumbent fail. He describes how big companies sustain their growth by improving their products and performances. Often this is done in an incremental way, little steps at the time to fine tune and do better what they do good. It can be in a radical way as well. In his book he shows how the hard disk drive industry got disrupted multiple times. Could this also be the case for national airlines like BA and KLM are they disrupted by the so called low cost carriers like Ryanair?

According to Christensen \cite{Christensen97} there are 3 elements causing disruption:
Firstly, big companies want to better their performance. This can be done by innovation but has as goal to make a better product for the customer. Disruptive technology are often new entrants that look for a new fringe market to sell a inferior product often cheaper or simpler or both than the original product. Targeting a niche-market with less demanding customers. Secondly, the incumbents in their attempt to improve the product overshoot the mainstream market, whereas the new entrants have been improving their cheaper and simpler product and move into the mainstream market.
And thirdly, the main industry can't react to the disruption because the entrants target as a small market with low margins. This is of no interest to the big companies. They need to grow and make more profit, for that they prefarble need higher margins. Furthermore, their customers demand a high quality product, they do not want a simpler product and often inferior product.

Figure \ref{fig:graph1} illustrates how this works. The big company is sustaining the technology and preformes better over time. So much so that they leave the mainstream and end up in the high end market. At the same time the new entrant is improving its product and enters the mainstream in time to take over from the incumbenmt \cite{Christensen97}.

\begin{figure}[h!]
    \centering
    \includegraphics[width=0.5\textwidth]{350.png}
    \caption{The figure from \cite{Christensen97}.}
    \label{fig:graph1}
\end{figure}


How do these elements hold up when applied to the airline business and Ryanair in particular.


\subsection{Ryanair as a entrant}
\label{sec:this-is-a-section}


The Incumbents in this case are national airlines like BA and KLM. On their flights they provided food and unlimeted drinks, allocated seats, generous bagage allowence and overall good service. Flights leave from prime location and main easy to reach airports. Flights leave on prime times for a high price.They want to better their preformance. \cite{Christensen97} than contineous to describe how this can be disrupted by newcomers that offer a product of worse quality than the incumbent. Although in in the beginning not seen as a treath because of the inferieur design or performance it can sometimes turn out to be the cause of big firms failure.

Looking at Ryanair's product \cite{Barrett} notes that Ryanair dropped cutsomer service items compared to the National European Airlines. No extra's on the flight and more seat per aircraft. Another major difference is the airport service, by using secondary airports like London Luton instead of Heatrow.These airports charge lower landing charges. A third obeservation by \cite{Barrett} is the fact that tickets are sold directly to the customer. "On their first flight customers paid in cash, which was colleted in a bucket as customers departed" \citep{ITBberlin}.

An other point is low maintaince becaue use of a single air craft (needs citation)

Instead of entering the regular market the Low Cost Airlines looked for new markets at the fringe. This is consitent with \cite{Christensen} first element.

The main airlines kept doing what they do best and maybe even overshoot the mainstream market by adding more and more 'frills'. New users enter the market and started flying with Low cost carries like Ryanair and their market grew, these travellers talked about this 'good enough' flight and also the regular existing market including business travellers heard about it and took an interest in these 'cheaper' but good enough flights, especially on short haul flights \citep{TiddBessant}.

Fare reduction was the first step to enter a new fringe market but they attracted other regular customers as well and not only becayse of the 'good enough' flights. Additionally, there was also improvements in the service that attracted people. Accoding to \cite{Barrett} Ryanair loses less bags and their arrivals are on time, on these service levels they score better than national carriers like BA or Air Francein 2004. He \citep{Barrett} than continues to explain that this is due to the fact that ryanair offers a simple product, no connecting flights and smaller airports, which makes it easier to meet these levels of service. Furthermore the secondary airports have other appaels like shorter walking distance and less congestion.

As explained by \cite{chistensen} the next step would be that the incumbent overshoots the target and the entrants move into the mainstream market.




\subsection{Ryanair moving into mainstream?}
\label{limits}

WEre quick with the internet..
No still low cost no frill some moved but not mainstream
no longhaul

also reaching bounderies


\subsection{Incumbant can't react}
\label{What did they do}

copying
paying for extra's
internet


\section{Another Section}




\label{sec:this-is-a-section}

\section{Conclusion}


100 woorden

The limits to disruption and how to make things cheaper are now entering the Human reaear departemnte and Bad publicity on how they treat personel make it unpopular there are many alternatives now.



You can see in Section \ref{sec:this-is-a-section} on page \pageref{sec:this-is-a-section} that things are not going well here.




\subsection{This is a subsection}
\label{sec:this-is-a-section}

%% disable some things
\renewcommand{\textbf}{}
\renewcommand{\bf}{}
\bibliography{biblio}{}
\end{document}
