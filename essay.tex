\documentclass[a4paper, 11pt]{article}

\usepackage{natbib}
\bibliographystyle{agsm}
\setcitestyle{authoryear,open={(},close={)}}

\input{layout.tex}

\title{\textbf{Essay}\\Sub title on Essay}

\author{\textsc{Ans Vaessen}
\\{\textit{ans@nlgids.london}}}

\date{\today}

%%%%%%%%%%%%%%%%% START %%%%%%%%%%%%%%%%%%

\begin{document} 
\maketitle 

\begin{abstract}
This is an abstract.
\end{abstract}

\vspace{30pt} % Some vertical space between the abstract and first section

\section*{Introduction}

Flying used to be an expensive mode of travel untill the low cost airlines entered the market. until in 1995 two airlines came with a new concept: Low Cost Airlines. This was an innovation that changed the airline business as we know it. 


See \cite{Christensen97}.
See \citep{Christensen97}.
See \citep[p. 145]{Christensen97}.

Also see this website I found: \cite{TripAdvisor}.
:
\begin{quote}
hallo
\end{quote}

\section{Formatting Citations}
In his book The Innovater Dillema \cite{Christensen97} describes how big companies sustain their growth by improving their products and preformances. Often this is done in an incremental way, little steps at the time to fine tune a do better what they do good. For example airlines like BA and KLM provided food and unlimeted drinks, allocated seats, generous bagage allowence and overall good service. FLying from prime location on prime times for a high price.\cite{Christensen97} than contineous to describe how this can be disrupted by newcomers that offer a product of worse quality than the incubent. Although in in the beginning not seen as a treath because of the inferieur design or performance it can sometimes turn out to be the cause of big firms failure.  
In the airline business for example Ryanair and Easy Jet. They are flying from smaller less fancy airports with no allocated seats, no free drinks or food and you pay for all the extra's. Their is no service. They will charge you for food, drinks, bagage, priority acces to the airplane etcetra. 
Instead of entering the regular market the Low Cost Airlines looked for new markets at the fringe. New users entered the market and it grew, the users talked about it and also the regular existing market came to hear about it and took an interest in these 'cheaper' but good enough flights \citep{TiddBessant}. 


Both started differnetly


\subsection{This is a subsection}
\label{sec:this-is-a-section}

\section{Another Section}
 used to be regulated by countries, each country had their own national airline. British Airways (BA) and (KLM) for the Netherlands for example. Goverments would agree between nations on flying schedules, amount of passengers and fares. There was no competition until the European Commision introduced a reform proces that changed the rules. "Since April 1997, any airline holding a valid Air Operators Certificate in the EU can operate on any route within the European Union, including wholly within another country, without restriction on price or capacity. As a result, European air travel has been flooded with an influx of low-cost airlines \citep{Eurocontrol}.

You can see in Section \ref{sec:this-is-a-section} on page \pageref{sec:this-is-a-section} that things are not going well here.

\bibliography{biblio}{}
\end{document}
